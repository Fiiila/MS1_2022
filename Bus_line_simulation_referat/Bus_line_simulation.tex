\documentclass{article}

\usepackage[dvipsnames]{xcolor}%barevne rozliseni kodu
\usepackage{mathptmx}
%packages for language
\usepackage[czech]{babel}
\usepackage[utf8]{inputenc}
%packages for graphic
\usepackage{graphicx}
\usepackage{pgfplots}
\usepackage{pgfplotstable}
\usepackage{multirow}
\usepackage{tikz}
\usepackage{import}
\usepackage[a4paper, total={17cm,25.7cm}, top=2cm, left=2cm, includefoot]{geometry}
\usepackage{todonotes}
\usepackage{standalone}
\usepackage{colortbl}%pro barevne zmeny v tabulce
\usepackage{float}
\usepackage{csvsimple} %pro import a práci s csv soubory
\usepackage{indentfirst}  % odsazení prvního řádku v odstavci
\usepackage{hyperref} %dela odkazy na mista v dokumentu atd
\usepackage{amsmath}%psani matic
\usepackage{mathrsfs}%psani kroucenym matematickym pismem
\usepackage{pdfpages}%vkladani celych pdf dokumentu
%cesta k obrazkum: ./Graphics/....
\usepackage{listings,lstautogobble}

\definecolor{mygreen}{rgb}{0,0.6,0}
\definecolor{mygray}{rgb}{0.5,0.5,0.5}
\definecolor{mymauve}{rgb}{0.58,0,0.82}

\lstset{ %
	backgroundcolor=\color{white},   % choose the background color
	basicstyle=\footnotesize,        % size of fonts used for the code
	breaklines=true,                 % automatic line breaking only at whitespace
	captionpos=b,                    % sets the caption-position to bottom
	commentstyle=\color{mygray},    % comment style
	escapeinside={\%*}{*)},          % if you want to add LaTeX within your code
	keywordstyle=\color{mymauve},       % keyword style
	stringstyle=\color{mygreen},     % string literal style
	numbers=left,
	autogobble=true,
}

\begin{document}
	\begin{titlepage}
    \begin{center}
        \LARGE
        Západočeská Univerzita v Plzni\\
        Fakulta Aplikovaných Věd\\
        
        \vspace{1cm}
        
        \includegraphics[width=0.5\textwidth]{./Graphics/FAV_logo.pdf}
        
        \vspace{4cm}
        
        \textbf{Bus Line Simulation}
        
        \vspace{0.5cm}
        Filip Jašek
        
    \end{center} 
    \vfill
        \noindent
        \large
        Předmět: KKY/MS1 (Modelování a simulace)\\
        Vyučující: Ing. Hajšman Václav, Ph.D., Ing. Liška Jindřich, Ph.D., Ing. Janeček Petr, Ph.D.\\
        Cvičící: Ing. Fetter Miloš\hfill Datum: \today
\end{titlepage}
	
	
	\section{Úvod a motivace}
		Pro svou semestrální práci jsem si vybral simulaci autobusové linky. Simulace byla vytvořena za pomoci balíčku "javaSimulation", který usnadňuje modelaci procesů v jazyce Java.\\
		
		Téma práce bylo zpracováno z pohledu tvorby simulace a nástroje pro provozovatele hromadné dopravy ve městech. Po mírných úpravách např. odlišných časů mezi zastávkami lze použít i pro dálkové autobusové linky. Hlavním smyslem práce však bylo vytvořit simulaci, která by poskytovala informace pro dopravce a dokázala poskytnout odpovědi na požadovanou kapacitu autobusů nebo jejich frekvenci a následnou změnu jízdních řádů.
	\section{Vypracování}
		Před započetím programování cílového řešení bylo potřeba ujasnit si, jaké třídy budou pro zvolenou simulaci potřeba. K tomu posloužila následující dekompozice procesu provozu autobusové linky.\\
		Třídy:
		\begin{itemize}
			\item \verb|BusLineSimulation|
			\item \verb|Bus|
			\item \verb|BusGenerator|
			\item \verb|BusStop|
			\item \verb|Passenger|
		\end{itemize}
		\subsection{BusLineSimulation}
			Tato třída slouží jako hlavní třída, v jejímž rámci probíhá samotná simulace. Kromě globálních parametrů pro celou simulaci a její analýzu obsahuje v metodě \verb|actions()| cyklus pro počáteční inicializaci, vytvoření odpovídajícího počtu zastávek a vložení do seznamu \verb|busLine|. Následně aktivuje generátor autobusů s předem definovanými parametry (\verb|bus_capacity|, čas po kterém autobusy vyjíždějí apod.). Po nastavené délce simulace \verb|simPeriod| v minutách a času navíc pro dokončení. Po doběhnutí simulace se do konzole vypíše výstup obsahující informace o průběhu a zakončení.
		\subsection{Bus}
			Třída pro reprezentaci autobusu a simulaci přepravy cestujících. Obsahuje informace o své kapacitě (\verb|capacity|), aktuální zastávce (\verb|busstopnumber|) a cestujících v seznamu \verb|passengers|. Obsahuje navíc ještě seznam \verb|getting_off|, do kterého se po naplnění autobusu zařadí cestující, co budou vystupovat na příští zastávce. Tento seznam není vyloženě nutný, ale v simulaci představuje lidi, kteří se postaví ke dveřím, aby co nejrychleji vystoupili.\\
			Autobus při obsluze pracuje ve třech fázích. V první nechá vystoupit cestující, kteří chtějí vystoupit. Ve druhé nechá nastoupit lidi na aktuální zastávce, dokud to kapacita autobusu dovolí a nakonec proběhne zařazení vystupujících do seznamu \verb|getting_off| a následně je autobus přemístěn po určeném čase na další zastávku.
		\subsection{BusGenerator}
			Jak již název napovídá, tato třída má na starosti generování autobusů v pravidelných intervalech podle jízdního řádu dokud neuplyne simulační čas. V cyklu vypouští nové autobusy na první zastávku, odkud pokračují na další zastávky v seznamu \verb|busLine|.
		\subsection{BusStop}
			Reprezentuje zastávku, kam chodí lidé s cílem přepravit se na jednu z následujících zastávek. Zastávka s určitým nadhledem funguje jako generátor cestujících a byla tak i naprogramována. Po zařazení zastávky do seznamu \verb|busLine| v počáteční inicializaci se spustí vnitřní cyklus, který začne generovat cestující do seznamu \verb|waitingPassengers| reprezentující frontu na zastávce. Tuto frontu obsahuje každá vygenerovaná zastávka a z ní si pak autobus nabírá své cestující.
		\subsection{Passenger}
			Jednoduchá třída odděděná od třídy \verb|Link|, která neobsahuje žádné aktivní procesy. Každý cestující má však jeden parametr, jímž je výstupní zastávka, která mu je však náhodně přiřazena zastávkou. Náhoda je omezena s ohledem na nástupní zastávku a celkový počet zastávek.
	\section{Realizace}
		Výstupem simulace je následující text, který informuje o jejím výsledku. V první části se lze dozvědět informace o počtu autobusů, zastávek a cestujících, kteří byli nebo nebyli přepraveni ze zastávek během simulace. Následuje výpis nepřepravených cestujících stále čekajících na autobus na jednotlivých zastávkách.\\
		Dále výstup obsahuje statistické informace o průměrném využití kapacit autobusů mezi jednotlivými zastávkami a časové srovnání proběhlé simulace a očekávaného ideálního času. Pro optimalizaci volby nových parametrů jsou vypsány i procentní využití času pro nástup, výstup a zpoždění.\\
		Pro ukázkovou simulaci byly použity jak následující parametry:
			\begin{itemize}
				\item počet zastávek - 5
				\item kapacita autobusu - 10
				\item simulační čas - 12*60 min
				\item ideální čas mezi zastávkami - 10 min
				\item interval mezi příjezdy autobusů na první zastávku - 90 min
				\item generování zpoždění, chyb, délky nástupů/ výstupů atd. - náhodně
			\end{itemize}
		
			\begin{verbatim}
				Simulation output...
				
				GENERAL SIMULATION INFO
				-----------------------
				Generated buses: 9
				Number of bus Stops: 5
				Number of generated passengers: 273
				Number of travelled passengers: 133
				Number of passenger on 1. bus Stops: 8
				Number of passenger on 2. bus Stops: 52
				Number of passenger on 3. bus Stops: 41
				Number of passenger on 4. bus Stops: 39
				Number of passenger on 5. bus Stops: 0
				
				SIMULATION STATISTICS
				---------------------
				Bus capacity:10
				Average used bus capacity between 1. and 2. stop is: 70.0 %
				Average used bus capacity between 2. and 3. stop is: 91.11111111111111 %
				Average used bus capacity between 3. and 4. stop is: 97.77777777777777 %
				Average used bus capacity between 4. and 5. stop is: 96.66666666666667 %
				Ideal travel time thru busline with 1 min for stopping and boarding passengers: 44.0 min
				Average bus travel time: 57.879226488546834 min
				Boarding takes: 3.032841858605662 %
				Exitting takes: 2.25373651158541 %
				Traffic jam takes: 4.974542099515745%
				Average time of boarding 1 passenger takes 0.11878547876267986 min
				Average time of exiting 1 passenger takes 0.08827073187940108 min
			\end{verbatim}
		\vspace{5mm}
		\noindent
		Na následujících stranách v příloze na straně \pageref{priloha1} je možné prohlédnou dříve popsaný zdrojový kód generující výstup výše.
	\section{Závěr}
		V této semestrální práci jsem si vyzkoušel jak se vytváří simulace reálného procesu a začal přemýšlet více o tom, jak by se i jiné procesy daly dekomponovat a simulovat, aby se bez testování dokázaly odhadnout potřebné parametry simulace.\\
		Simulace autobusové linky se povedla přesně podle očekávání a pracuje tak jak má. Analytická vygenerovaná data odpovídají očekávání a dokážou zpřesnit představu o fungování reálného procesu. Jedná se však o velmi zjednodušený simulační model, ale dalo by se ho jednoduše rozšířit o další data nebo náhodné chyby či rozdílné generování lidí podle času během dne apod.\\
		Závěrem považuji semestrální práci za zdařilou, splňující očekávání.
		\newpage
	\section{Příloha - 1}\label{priloha1}
			\begin{lstlisting}[language=java]
				import javaSimulation.*;
				import javaSimulation.Process;
				
				public class BusLineSimulation extends Process{
					/**
					* class representing bus line
					* by calling constructor of this class you will receive data related to your parameters
					*/
					int noOfBusStops;//number of bus stops in the busline
					Head busLine = new Head();//queue contains bus stops
					Random random = new Random(5);
					//variables for simulation analysis
					int generated_passengers;//total number of generated passengers on busstops
					int travelled_passengers;//number of transported passengers
					int generated_buses;//number of generated buses in simulation time
					double[] avg_used_bus_capacity;//average used bus transport capacity between bus stops
					double realTravelTime;
					double boardingTime;
					double exitTime;
					//variables for changing simulation environment
					int bus_capacity;//capacity of generated buses
					double simPeriod = 12*60;//simulation time
					double idealTravelTime=10;//ideal travel time between two bus stops
					
					BusLineSimulation(int n, int capacity) {noOfBusStops = n;avg_used_bus_capacity = new double[n];bus_capacity=capacity;}
					
					public void actions() {
						//running simulation
						for (int i = 1; i <= noOfBusStops; i++)
						activate(new BusStop());//generating bus stops based on given parameter
						activate(new BusGenerator());//activating bus generator
						//hold for simulation time plus some added time to let simulation finish it work
						hold(simPeriod+1000000);
						report();//generate report about simulation
					}
					
					void report() {
						System.out.println("\nSimulation output...");
						System.out.println("Number of bus Stops: " + busLine.cardinal());
						//printing all individual passengers and printing their entry and exit stop
						Link stoplink = busLine.first();
						BusStop stop;
						for (int i = 1; i <= noOfBusStops;i++){
							stop = (BusStop)stoplink;
							System.out.println(("Number of passenger on "+ i +". bus Stops: " + stop.waitingPassengers.cardinal()));
							stoplink = stoplink.suc();
						}
						//statistics based on fullness of busses
						System.out.println("Number of generated passengers: "+generated_passengers);
						System.out.println("Number of travelled passengers: "+travelled_passengers);
						System.out.println("Generated buses: "+generated_buses);
						for(int i=0;i<noOfBusStops-1;i++){
							System.out.println("Average used bus capacity between "+ (i+1) + ". and " + (i+2)+ ". stop is: "+ 100*avg_used_bus_capacity[i]/(bus_capacity*generated_buses) + " %");
						}
						System.out.println("Ideal travel time thru busline with 1 min for stopping and boarding passengers: "+ ((idealTravelTime+1)*(noOfBusStops-1))+" min");
						System.out.println("Average bus travel time: " + (realTravelTime/generated_buses)+" min");
						System.out.println("Boarding takes: " + (100*boardingTime/realTravelTime) +" %");
						System.out.println("Exitting takes: " + (100*exitTime/realTravelTime)+" %");
						System.out.println("Traffic jam takes: " + (100*(realTravelTime-((idealTravelTime+1)*generated_buses*noOfBusStops))/realTravelTime)+"%");
						System.out.println("Average time of boarding 1 passenger takes "+ (boardingTime/travelled_passengers)+" min");
						System.out.println("Average time of exiting 1 passenger takes "+ (exitTime/travelled_passengers)+" min");
						
						
						
					}
					
					
					
					class Bus extends Process{
						/**
						* class representing bus in bus line simulation
						*/
						public void actions(){
							//internal parameters specifying bus
							int capacity = bus_capacity;//
							int busstopnumber = 1;//actual number of busstop
							Head passengers = new Head();//passengers in bus capacity
							Head getting_off = new Head();//passengers exiting next stop (waiting near doors)
							Link busstop_link = busLine.first();//link onto first bus stop
							//variables for helping with processing
							BusStop busstop;
							Link passengerlink;
							Passenger passenger;
							//statistics parameters
							double inTime;
							double totalInTime=0;
							double outTime;
							double totalOutTime=0;
							double start = time();//bus starting time
							//cycle what bus do from time its generated till final stop
							while(busstopnumber<=noOfBusStops) {
								//exiting
								while(!getting_off.empty()){
									getting_off.first().out();//unboarding / exiting
									outTime = random.uniform(0.0,15.0)/60;//generating random exit time
									totalOutTime+=outTime;
									hold(outTime);//time that passenger needs to exit bus
								}
								//boarding
								busstop = (BusStop) busstop_link;
								while(passengers.cardinal()<capacity) {
									if(!busstop.waitingPassengers.empty()){
										passengerlink = busstop.waitingPassengers.first();
										passenger = (Passenger)passengerlink;
										passengerlink.out();
										passenger.into(passengers);
										travelled_passengers+=1;
										inTime = random.uniform(0.0,15.0)/60;//generating random exit time
										totalInTime+=inTime;
										hold(inTime);//one passenger boarding time
										continue;
									}else {
										break;
									}
								}
								//counting passengers in bus for statistics
								if(busstopnumber<noOfBusStops){
									avg_used_bus_capacity[busstopnumber-1] += passengers.cardinal();
								}
								
								//nalezeni pasazeru, co vystupuji nasledujici zastavku
								passengerlink = passengers.first();
								passenger = (Passenger)passengers.first();
								for(int i=0 ;i<passengers.cardinal();i++){
									if(busstopnumber==noOfBusStops-1){
										passenger.out();
										passenger.into(getting_off);
										passengerlink = passengers.first();
										passenger = (Passenger)passengerlink;
									}
									if(passenger.exitStop==busstopnumber+1){
										passenger.out();
										passenger.into(getting_off);
										passengerlink = passengers.first();
										passenger = (Passenger)passengerlink;
										continue;
									}
									passengerlink = passengerlink.suc();
									passenger = (Passenger)passengerlink;
								}
								busstop_link = busstop_link.suc();
								busstopnumber += 1;
								hold(idealTravelTime+random.uniform(0.0,2.0));//travel time plus delay
							}
							//counting complete statistics
							boardingTime+=totalInTime;
							exitTime+=totalOutTime;
							realTravelTime+=time()-start;
						}
					}
					
					class BusGenerator extends Process{
						/**
						* class representing some depo from which drive buses in periodic time given by bus line schedule
						*/
						public void actions(){
							//generating busses in fixed interval
							while(time()<=simPeriod){
								activate(new Bus());
								generated_buses+=1;
								hold(90);//waiting time between bus departures
							}
						}
					}
					
					class Passenger extends Link{
						/**
						* class representing passenger with his own exit stop
						*/
						int exitStop;
						Passenger(int e){exitStop=e;}
					}
					
					class BusStop extends Process{
						/**
						* class representing bus stop
						* passengers appears here so it works like passenger generator
						*/
						Head waitingPassengers = new Head();//waiting passengers on bus stop
						int no_of_stop;//process variable holding number of this bus stop
						public void actions(){
							into(busLine);//taken into bus line
							no_of_stop = busLine.cardinal();//given number based on position in bus line
							if(no_of_stop<noOfBusStops){
								//generating passengers on busstop
								while (time() < simPeriod) {
									new Passenger(random.randInt(no_of_stop+1,noOfBusStops)).into(waitingPassengers);//new waiting passenger
									generated_passengers +=1;//statistics of total generated passengers
									//statistics about generated passengers and their desired exit stop
									System.out.println("pasazer cekajici na zastavce: "+no_of_stop+" vystupuje na zastavce: " + ((Passenger)waitingPassengers.first()).exitStop);
									hold(random.negexp(1/11.0));//hold time defining frequency of generating passengers
								}
							}
							
							
						}
					}
					
					public static void main(String args[]){
						activate(new BusLineSimulation(5,10));
					}
				}
				
			\end{lstlisting}
	
\end{document}
